\documentclass[a4paper,14pt]{article}
\usepackage{extsizes}		% 14 шрифт
\usepackage{ucs}			% Включаем поддержку UTF8
\usepackage[utf8x]{inputenc}% Включаем поддержку UTF8
\usepackage[russian]{babel} % Включаем пакет для поддержки русского языка
\providecommand{\No}{№} 	% для сборки с babel >= 1.2 
\usepackage{indentfirst}    % Начинать первый параграф с красной строки
\setlength{\parindent}{1cm} % Отступ
\bibliographystyle{gost780u}% стиль библиографии
\usepackage{amstext} 		% кирилица в формулах
\usepackage{lscape}			% альбомная ориентация листа
\usepackage{textcomp} 		%Команды для специальных символов в текстовом режиме
\uchyph=0					% Запрет переноса слов с прописной буквы
\usepackage{multirow}
\usepackage{uspd}
\usepackage{mydoc}
\usepackage{float}

\makeindex 
\title{\textbf{СПЕЦИАЛЬНОЕ ПРОГРАММНОЕ ОБЕСПЕЧЕНИЕ «КАМА-НАДИР»} \\Описание программы}
%\def\year{2016}
\uspdabstract{В документе описаны назначение СПО «Кама-Надир», средства его реализации, требования к аппаратному и программному обеспечению, 
необходимые для устойчивой работы программы и иные смежные вопросы, имеющие первостепенное значение.

В разделе \ref{logic} структура программы описана в привязке к решаемым задачам, 
приведены полные или же упрощенные схемы алгоритмов их решения и взаимодействия между ними, 
а также описаны массивы входных и выходных данных по каждой решаемой задаче.}

\begin{document}
\begin{uspd}{ХХХХ.ХХХХХ-ХХ~ХХ~ХХ}
\section{Функциональное назнчение} \label{purpose}
\subsection{Назначение СПО «Кама-Надир»}
СПО «Кама-Надир» представляет собой встраиваемое программное обеспечение, предназначенное для обработки информации от ИИБ 12.002, НАП ГНСС, 
лага и выработки на их основании навигационных параметров и параметров ориентации объекта, 
реализованных в соответствии с переданными Заказчиком алгоритмами.
\subsection{Общее описание функционирования программы}
\begin{itemize}
    \item программа работает под управлением операционной системы реального времени QNX-6.5.0;
    \item программа принимает данные от ИИБ-12.002, НАП ГНСС, лага (цифрового или импульсного);
    \item Обрабатывает и ассоциирует данные, выполняет проверку годности принятых данных;
    \item Далее программа реализует навигационный цикл, в соответствии с  блок-схемой на Рисунок 1.
    \item дополнительно программа выполняет контроль и статусы периферийного оборудования, взаимодействие с пультом оператора ПО5, 
    реализуя заложенные в него функции, функции расширенного контроля принимаемых от НАП ГНСС данных;
    \item результаты вычислений транслируются потребителям: ПО5, канал RS422 (внешний потребитель), 
    канал реального времени Manchester, межканальный обмен с параллельным каналом.
\end{itemize}
\subsection{Требования к программному обеспечению}
Программа предназначена для функционирования под управлением ОС реального времени (ЗОС РВ «Нейтрино», 
QNX-6.5.0, Debian Buster, Raspberry Pi OS (ранее Raspbian), 
MOXA Industrial Linux, Debian Stretch).
\subsection{Требования к аппаратной платформе}
Работа программы проверялась на следующих аппаратных платформах: x86, ARM (Cortex-A8), RISC.
\subsection{Структура программы и ее составные части}
Основными составными частями СПО «Кама-надир» являются:
\begin{itemize}
    \item /nadir/bin/nadir - исполняемый модуль;
    \item /nadir/bin/cpc - драйвер счетчика импульсов аналогового лага;
    \item /nadir/lib/libcpcapi.a - библиотека взаимодействия с драйвером счетчика импульсов аналогового лага;
    \item /nadir/lib/libkernel.so - библиотека базовой функциональности;
\end{itemize}
Плагины:
\begin{itemize}
\item /nadir/lib/libstdthread.so -  реализация потоков выполнения процессоров данных;
\item /nadir/lib/libdpexchangeng.so - реализация процессора данных внутреннего обмена данными;
\item nadir/lib/libdpparserchain.so - реализация процессора данных цепочки декодирования входной информации и кодирования выходной;
\item /nadir/lib/libdpsync.so - реализация процессора данных синхронизации вычислителей;
\item /nadir/lib/libdpalignment.so - реализация процессора данных основного алгоритма;
\item /nadir/lib/libdptime.so - реализация процессора данных установки системного времени;
\item /nadir/lib/libdpkamatmk.so - реализация процессора данных передачи информации по протоколу ИТС №5;
\item /nadir/lib/libdprmcanalyser.so - реализация процессора данных анализатора принятых сентенций RMC;
\item /nadir/lib/libdpmodectrl.so - реализация процессора данных обработчика переключения режимов работы (сервисный/нормальный);
\item /nadir/lib/libdpsleep.so - реализация процессора данных задержки обработки входных данных;
\item /nadir/lib/libethiface.so - реализация сетевых интерфейсов сопряжения;
\item /nadir/lib/libserialiface.so - реализация последовательного интерфейса сопряжения;
\item /nadir/lib/libpliface.so - реализация интерфейса сопряжения с аналоговым лагом;
\item /nadir/lib/libsyncparser.so - реализация кодирования/декодирования данных синхронизации вычислителей;
\item /nadir/lib/libiibparser.so - реализация декодирования данных ИИБ;
\item /nadir/lib/libpo5parser.so - реализация кодирования/декодирования данных пульта оператора (протокол ИТС №101);
\item /nadir/lib/libnmeaparser.so - реализация декодирования навигационных данных принятых по протоколу ИТС IEC 61162-1 ed. 4.0;
\item /nadir/lib/libconsumer.so - реализация кодирования данных "потребителя" (протокол ИТС IEC 61162-1 ed4.0);
\item /nadir/lib/libregistrator.so - реализация кодирования данных "регистратора" (протокол ИТС №100);
\item /nadir/lib/libexhibitorparser.so - реализация кодирования/декодирования данных технологического ПО "кама-терминал";
\item /nadir/lib/libplparser.so - реализация кодирования/декодирования данных аналогового лага.
\end{itemize}
\subsection{Язык программирования}
Код программы написан на языках прораммирования С++`14, С`11.
Используемые библиотеки: stdlib, libboost.
\section{Логика работы программы} \label{logic}
\subsection{Структурирование программы по Задачам}
Работа СПО «Кама-Надир» структурирована по решаемым задачам согласно  схеме на Рис.~\ref{fig:general_scheme}
\begin{figure}[H]
    \centering
    \includegraphics{images/general_scheme.png}
    \label{fig:general_scheme}
\end{figure}
\subsection{Задача формирования сигналов FS}
Реализует следующие функции согласно схеме на
\begin{figure}[H]
    \centering
    \includegraphics{images/FS(TO1n).png}
    \label{fig:general_scheme}
\end{figure}
\begin{itemize}
\item Принимает от задачи VOG сигналы - q1,q2,q3,w1,w2,w3 и формирует с учетом принятой модели  инструментальных погрешностей передаваемые в задачи PK и PS  
приращения угла поворота qx1,qy1,qz1 и кажущейся скорости wx1,wy1,wz1 в проекциях на оси БЧЭ.
\item Преобразует сигналы горизонтных каналов ВОГ- q1,q2  к осям обьекта при значении признака ориентации POR=1 в случае установки корпуса БЧЭ 
с поворотом на 180 относительно продольной оси объекта.
\item Осуществляет масштабирование, компенсацию аддитивных и мультипликативных составляющих модели  инструментальных погрешностей   
сигналов ВОГ и акселерометров с использованием задаваемых в случае необходимости в файле данных KAM.DAT корректур, а также меняющихся в запуске 
и  оцениваемых оптимальным фильтром Калмана ( ОФК ) составляющих дрейфов в осях БЧЭ:  
    \begin{itemize}
\item систематических ошибок  pm1a, qm1a, rm1a,wx0, wy0, wz0
\item масштабных коэффициентов kx, ky, kz,  kwx, kwy, kwz
\item невыставок  dxy, dxz, dyx, dyz, dzx, dzy, dаxy, dаxz, dаyx
\item оценки дрейфов dpf, dqf, drf
    \end{itemize}
\end{itemize}

\section{Схема алгоритма работы СПО «Кама-Надир»}
Файл приложения.
\end{uspd}
\end{document}
