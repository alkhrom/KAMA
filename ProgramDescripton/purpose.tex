\section{Функциональное назнчение} \label{purpose}
\subsection{Назначение СПО «Кама-Надир»}
СПО «Кама-Надир» представляет собой встраиваемое программное обеспечение, предназначенное для обработки информации от ИИБ 12.002, НАП ГНСС, 
лага и выработки на их основании навигационных параметров и параметров ориентации объекта, 
реализованных в соответствии с переданными Заказчиком алгоритмами.
\subsection{Общее описание функционирования программы}
\begin{itemize}
    \item программа работает под управлением операционной системы реального времени QNX-6.5.0;
    \item программа принимает данные от ИИБ-12.002, НАП ГНСС, лага (цифрового или импульсного);
    \item Обрабатывает и ассоциирует данные, выполняет проверку годности принятых данных;
    \item Далее программа реализует навигационный цикл, в соответствии с  блок-схемой на Рисунок 1.
    \item дополнительно программа выполняет контроль и статусы периферийного оборудования, взаимодействие с пультом оператора ПО5, 
    реализуя заложенные в него функции, функции расширенного контроля принимаемых от НАП ГНСС данных;
    \item результаты вычислений транслируются потребителям: ПО5, канал RS422 (внешний потребитель), 
    канал реального времени Manchester, межканальный обмен с параллельным каналом.
\end{itemize}
\subsection{Требования к программному обеспечению}
Программа предназначена для функционирования под управлением ОС реального времени (ЗОС РВ «Нейтрино», 
QNX-6.5.0, Debian Buster, Raspberry Pi OS (ранее Raspbian), 
MOXA Industrial Linux, Debian Stretch).
\subsection{Требования к аппаратной платформе}
Работа программы проверялась на следующих аппаратных платформах: x86, ARM (Cortex-A8), RISC.
\subsection{Структура программы и ее составные части}
Основными составными частями СПО «Кама-надир» являются:
\begin{itemize}
    \item /nadir/bin/nadir - исполняемый модуль;
    \item /nadir/bin/cpc - драйвер счетчика импульсов аналогового лага;
    \item /nadir/lib/libcpcapi.a - библиотека взаимодействия с драйвером счетчика импульсов аналогового лага;
    \item /nadir/lib/libkernel.so - библиотека базовой функциональности;
\end{itemize}
Плагины:
\begin{itemize}
\item /nadir/lib/libstdthread.so -  реализация потоков выполнения процессоров данных;
\item /nadir/lib/libdpexchangeng.so - реализация процессора данных внутреннего обмена данными;
\item nadir/lib/libdpparserchain.so - реализация процессора данных цепочки декодирования входной информации и кодирования выходной;
\item /nadir/lib/libdpsync.so - реализация процессора данных синхронизации вычислителей;
\item /nadir/lib/libdpalignment.so - реализация процессора данных основного алгоритма;
\item /nadir/lib/libdptime.so - реализация процессора данных установки системного времени;
\item /nadir/lib/libdpkamatmk.so - реализация процессора данных передачи информации по протоколу ИТС №5;
\item /nadir/lib/libdprmcanalyser.so - реализация процессора данных анализатора принятых сентенций RMC;
\item /nadir/lib/libdpmodectrl.so - реализация процессора данных обработчика переключения режимов работы (сервисный/нормальный);
\item /nadir/lib/libdpsleep.so - реализация процессора данных задержки обработки входных данных;
\item /nadir/lib/libethiface.so - реализация сетевых интерфейсов сопряжения;
\item /nadir/lib/libserialiface.so - реализация последовательного интерфейса сопряжения;
\item /nadir/lib/libpliface.so - реализация интерфейса сопряжения с аналоговым лагом;
\item /nadir/lib/libsyncparser.so - реализация кодирования/декодирования данных синхронизации вычислителей;
\item /nadir/lib/libiibparser.so - реализация декодирования данных ИИБ;
\item /nadir/lib/libpo5parser.so - реализация кодирования/декодирования данных пульта оператора (протокол ИТС №101);
\item /nadir/lib/libnmeaparser.so - реализация декодирования навигационных данных принятых по протоколу ИТС IEC 61162-1 ed. 4.0;
\item /nadir/lib/libconsumer.so - реализация кодирования данных "потребителя" (протокол ИТС IEC 61162-1 ed4.0);
\item /nadir/lib/libregistrator.so - реализация кодирования данных "регистратора" (протокол ИТС №100);
\item /nadir/lib/libexhibitorparser.so - реализация кодирования/декодирования данных технологического ПО "кама-терминал";
\item /nadir/lib/libplparser.so - реализация кодирования/декодирования данных аналогового лага.
\end{itemize}
\subsection{Язык программирования}
Код программы написан на языках прораммирования С++`14, С`11.
Используемые библиотеки: stdlib, libboost.